\documentclass[10pt]{beamer}

\usetheme[progressbar=frametitle]{metropolis}
\usepackage{appendixnumberbeamer}
\usepackage[utf8]{inputenc} %Required for encoding text
\usepackage[T1]{fontenc}
\usepackage[french]{babel}
\usepackage{booktabs}
\usepackage[scale=2]{ccicons}

\usepackage{pgfplots}
\usepgfplotslibrary{dateplot}
\usepackage{xspace}

\usepackage{xcolor}
\usepackage{listings}
\usepackage{tcolorbox}
\tcbuselibrary{listings,skins}
%\lstset{basicstyle=\ttfamily\color{white},
%  showstringspaces=false,
%  xleftmargin = 0.5cm,
%  commentstyle=\color{red},
%  keywordstyle=\color{blue},
%  rulecolor=\color{white},
%  frameshape={RYR}{Y}{Y}{RYR},
%  backgroundcolor=\color{black},
%}
\lstdefinestyle{mystyle}{
     basicstyle=\ttfamily\color{white},
     language=bash,
     tabsize=1,
     keywordstyle=\color{blue}\bf,
     showstringspaces=false,
     morekeywords={mount, umount, ls, 
	 		cat, adduser, chown, chgrp, init, fuser}
 }
\newtcblisting{mylisting}{
      enhanced,                             %%% needed for shadow
      arc=2mm,
      top=0mm,
      bottom=0mm,
      left=0mm,
      right=0mm,
      boxrule=0pt,
      colback=black,
      %shadow={5mm}{-3mm}{0mm}{fill=black!50!white,opacity=0.5},             %%% here for shadow  and adjust as you like
      listing only,
      listing options={style=mystyle},
      hbox
}

\newcommand{\linux}{\textbf{\textsc{Linux}}\xspace}
\newcommand{\param}[1]{\textcolor{blue}{#1}}

\addtobeamertemplate{section page}{}{
  \centering
  \begin{minipage}{22em}
  \tableofcontents[sectionstyle=hide/hide,hideothersubsections]
  \end{minipage}\par
}
\title{Shell : commandes UNIX et programmation système}
\subtitle{Cours d'Outils de Programmation}
% \date{\today}
\date{}
\author{Bousbaa Imad Eddine}
\institute{Université des Sciences et de la Technologie de Houari Boumediène}
% \titlegraphic{\hfill\includegraphics[height=1.5cm]{logo.pdf}}

\begin{document}

\maketitle

\begin{frame}{Plan}
  \setbeamertemplate{section in toc}[sections numbered]
  \tableofcontents[hideallsubsections]
\end{frame}

\section{}

\begin{frame}[fragile]{Objectifs}
Apprendre à utiliser les commandes UNIX (bash).\\
Apprendre à automatiser des tâches en écrivant des scripts.\\

\end{frame}

\section{Shell}

\begin{frame}{Objectifs de shell}
\begin{enumerate}
\item Fournir une interface pour la saisie de commande
\item Redirection des entrées/sorties standards
\item Analyser les commandes
\begin{itemize}
	\item substitution de noms de fichiers
	\item substitution de variables
	\item redirection d’entrées/sorties
\end{itemize}
\item Exécution de commandes
\begin{itemize}
\item mode synchrone
\item mode asynchrone
\end{itemize}
\item Fournir un langage interprété
\end{enumerate}
\end{frame}


\begin{frame}{Type de shell}
\begin{itemize}
\item \alert{Bourne Shell}  \param{sh} : Shell disponible sur toute plateforme UNIX.
\item \alert{C shell}  \param{csh} : Shell développé par BSD.
\item \alert{Korn shell}  \param{ksh} : Bourne Shell étendu par l’AT\&T.
\item \alert{Bourne Again Shell} \param{bash} : Version améliorée de sh et csh. Fourni le plus souvent avec Linux.
\item \alert{Zero Shell}  \param{zsh} : shell avec beaucoup de fonctionnalités : typage, substitution et complétion très poussées.
\item \alert{Tenex} \param{tcsh}  : csh étendu
\item \alert{rc} \param{rc}  : Implémentation pour UNIX du shell de Plan 9
\end{itemize}
\end{frame}

\begin{frame}{Utilisation du shell}
Deux modes d’utilisation :
\begin{itemize}
\item interactif : en ligne de commande.
\begin{enumerate}
  \item Présente une invite (prompt) à l’utilisateur et attend que celui-ci tape une commande ;
  \item Exécute * la commande tapée par l’utilisateur
  \item Retour en $1$.
\end{enumerate}
\item non-interactif : scripts shell, batch
  \begin{enumerate}
  \item Lit une ligne du fichier
  \item Exécute * les instructions données dans la ligne du fichier
  \item Passe à la ligne suivante
  \item Retour en $1$
  \end{enumerate}
\end{itemize}

Le programme s’arrête lorsqu’il n’y a plus de ligne à lire ou lorsqu’un instruction spéciale (exit ou return) est rencontrée.

\end{frame}

\section{Bash}
\subsection{Les console}
\begin{frame}[fragile]{c'est quoi une console ?}

La console est une interface entre votre système et vous. Elle s'appuie sur un
interpréteur de commandes (shell en anglais), sorte de langage. Il en existe
plusieurs dans le monde UNIX mais le plus utilisé s'appelle Bash. Il est installé par
défaut sur probablement toutes les distributions Linux et contient des centaines de
commandes différentes.

\end{frame}


\begin{frame}[fragile]{Les consoles}

Bash peut être utilisé directement en ligne de commande, c'est-à-dire en dehors de
toute interface graphique (KDE, Gnome etc). Il existe aussi des interfaces graphiques. Il en existe de nombreuses : Gnome-Terminale, konsole, aterm, eterm, xterm... Leurs différences étant cosmétiques car elles utilisent finalement le même interpréteur de commande, Bash.

Vous accédez ainsi à Bash si, après le démarrage du noyau Linux, vous ne vous connectez pas à un environnement graphique (KDE, Gnome...). Sa puissance est alors à son maximum car le processeur ne s'occupe pas de gérer des fenêtres. 

Pour ouvrir une console, le plus simple est d'explorer le menu de l'environnement graphique à sa recherche ou bien de cliquer l'icône probablement installée dans la barre des tâches.
\end{frame}


\begin{frame}{Les terminaux virtuels}

Que vous soyez connecté à une interface graphique ou non, il existe un moyen
pratique d'ouvrir rapidement une ligne de commande supplémentaire. Le raccourci
clavier est CRTL-ALT-Fx. Remplacez le x par un chiffre de 1 à 7.

Exemple : \alert{CRTL-ALT-F1}. 

Les 6 premiers ouvrent des terminaux virtuels (auxquels il faudra vous connecter : identifiant et mot de passe) et le 7èm ramène à l'interface graphique (si ouverte). C'est surtout pratique quand aucune interface graphique (KDE, Gnome...) n'est ouverte, vous pouvez ainsi lancer en parallèle des commandes.
\end{frame}

\begin{frame}{Le prompt}
	Le prompt est une invite de commande. C'est le message que la console place à
chaque début de ligne, en attendant votre commande.

Exemple : \\
\alert{luteola@localhost \textasciitilde \$}


Ceci indique que le simple utilisateur (symbole \$) 'luteola' travaille sur l'ordinateur
'localhost' est qu'il est dans son répertoire personnel (symbole \textasciitilde) : /home/luteola.


\alert{root@localhost /bin \#}
Dans cet exemple, l'administrateur (root et symbole \#) travaille dans le dossier
/bin.
\end{frame}


\begin{frame}{Sensibilité à la casse}
La console est sensible à la casse (majuscule / minuscule) ! Ainsi, ls est une
commande alors que LS ne veut rien dire. Idem pour l'écriture des dossiers et
fichiers. La moindre faute ou espace mal placé aura des conséquences.
\end{frame}


\begin{frame}{Écrire des chemins d'accès.}
	La structure des répertoires sous Linux est pyramidale, la pointe étant appelée la
racine. 

En console, on peut bien sûr se déplacer dans ces répertoires.


Pour séparer des répertoires dans un chemin, on utilise des slashs /, comme on le
ferait pour une adresse Internet en fait. N'utilisez donc pas les anti-slash, comme
vous le demanderai Windows. 

Exemple : /home/george/musique.\pause

Les anti-slash servent plutôt à insérer (on dit 'protéger') des caractères spéciaux ou
des espaces. Créez 'à la souris' un répertoire "Nouveau Dossier" sur votre bureau
(clic droit, Nouveau...) puis en console tapez cd \textasciitilde/Desktop/ et utilisez la touche
Tabulation (qui autocomplète un chemin). 'Nouveau' et 'dossier' sont séparés par
un anti-slash.
\end{frame}


\begin{frame}{Autocomplétion des commandes}
	Bash peut "deviner" la fin d'un chemin ou d'une commande. Ceci s'opère par la touche 'tabulation'. Très pratique pour accélérer la saisie de commandes ou retrouver une syntaxe ! 
	
	Exemples :

	Tapez up et appuyez sur 'tabulation'. Bash affiche alors la liste des commandes commençant par 'up'.

	Tapez cd /bi puis 'tabulation'. Bash complète la fin du chemin s'il existe, cd /bin/ en l'occurrence.

	Mais encore plus fort avec "bash\_completion", il peut même compléter lors de l'installation de logiciels.

	Par exemple sur Ubuntu ou Debian si vous tapez "source /etc/bash\_completion" et qu'en suite vous tapez "aptitude install fire" puis la touche "tabulation", il va vous donné tous les logiciels disponibles qui commencent par "fire"
\end{frame}


\begin{frame}{Historique des commandes}

En utilisant les flèches haut et bas, vous parcourez l'historique des commandes tapées sous votre login.

Ou avec la commande "history" il peut vous retrouver des commandes que vous
avez tapez le matin ou la veille.


Exemple : \$ \alert{history} | \alert{grep} cat

\alert{CTRL-R} permet également de faire une recherche rapide sur une commande.
\end{frame}

\begin{frame}{jokers}
Vous pouvez utiliser des jokers dans une commande :
\begin{itemize}
\item ? représente n'importe quel caractère (a, Y, 3, @...).
\item * représente n'importe quel chaîne de caractères (un mot, une phrase...).
\item Exemple : ls photo* affiche la liste des fichiers ou dossiers commençant par
\item le mot photo.
\item *[a]* représente une chaîne contenant la lettre a.
\item *[!a]* représente une chaîne ne contenant pas la lettre a.
\item [abc]* représente une chaîne commençant par la lettre a,b ou c.
\item [a-d]* représente une chaîne commençant par la lettre a, b, c ou d.
\end{itemize}
\end{frame}
\section{Les principales commandes console (BASH) sous Linux}

\begin{frame}[fragile]{syntaxe d'une commande}
la syntaxe classique d'une commande Unix :

\begin{center}
cmd [-opt1] [-opt2] ... [--] arg1 arg2 ...
\end{center}

où :

\begin{itemize}
\item  "cmd" correspond au nom de la commande,
\item  "-opt" correspond à une option possible,
\item  "arg" correspond à un argument.
\end{itemize}
\pause

Notons que les arguments sont le plus souvent un nom de fichier que la commande manipule. On peut utiliser "--" pour séparer explicitement les options des arguments quand la commande est ambiguë.

Pour obtenir l'aide en ligne sur une commande, il suffit d'utiliser le "man". Pour obtenir l'aide sur la commande "cmd", il suffit de taper :
\begin{mylisting}
man cmd # Afficher le manuel de la commande cmd
\end{mylisting}
\end{frame}

\begin{frame}[fragile]{Type de commmande}
Les commandes internes sont réalisées de manière interne par le shell lui-même ; c’est-à-dire qu’il n’y a pas de création de processus
pour exécuter la commande. Ces commandes ne possèdent pas d’exécutables associés puisqu’elles sont codées en interne au shell.

Une méthode pour identifier les builtins est d’utiliser la commande interne type.
\begin{mylisting}
$ type cd
cd is a shell builtin
$ type echo
echo is a shell builtin
$ type ls
ls is /bin/ls
$ type cat
cat is /bin/cat
$ type gcc
gcc is /usr/bin/gcc
\end{mylisting}

\end{frame}

\begin{frame}{Commandes}
Voici une liste des commandes console Bash les plus utilisées en routine. Certaines
plus que d'autres. N'oubliez pas que souvent des interfaces graphiques existent
pour ces commandes.
\end{frame}
\begin{frame}{Commandes de base }
\begin{itemize}
\item \alert{ls} : Afficher la liste des dossier et fichiers d'un répertoire.
\item \alert{cd} : Se déplacer vers un répertoire.
\item \alert{pwd} : Affiche le nom du répertoire courant.
\item \alert{mkdir} : Créer un répertoire
\item \alert{rmdir} : Supprimer un répertoire.
\item \alert{rm} : Effacer un fichier ou un répertoire.
\item \alert{cp} : Copier un dossier ou un répertoire.
\item \alert{mv} : Déplacer (renommer) un dossier ou un répertoire.
\item \alert{ln} : Créer un lien vers un fichier.
\item \alert{touch} : Permet de changer la date d'un fichier à l'heure courante ou de créer un fichier s'il n'existe pas (touch /home/user/test).
\item \alert{alias} : Créer un alias d'une commande ( exemple : alias rm="rm -i" )
\end{itemize}
\end{frame}

\begin{frame}{Manipulations de texte}
\begin{itemize}
\item \alert{cat} : Afficher le contenu d'un fichier texte. Concaténer des fichiers.
\item \alert{more}, less : Afficher le contenu d'un fichier texte intelligemment.
\item \alert{grep} : Rechercher une chaîne de caractère.
\item \alert{sed} : Remplacer des occurrences de texte dans des fichiers.
\item \alert{echo} : Envoyer un message texte. Par exemple vers un fichier.
\item \alert{tail} : N’affiche que les dernières lignes d’un fichier (-n permet de spécifier le nombre de lignes à afficher).
\item \alert{head} : Comme tail, mais affiche les N premières lignes d’un fichier (N=10 par défaut).
\item \alert{sort} : Trier un fichier texte.
\item \alert{uniq} : Supprime ou compte les doublons sur un fichier trier.
\item \alert{wc} : Compte le nombre de lignes, mots et caractères.
\item \alert{awk} : Remise en forme de données texte.
\end{itemize}
\end{frame}

\begin{frame}{Installer des logiciels}
\begin{itemize}
\item \alert{urpmi} : Installer des paquetages RPM sous Mandriva.
\item \alert{apt-get}, \alert{aptitude} : Installer des paquets DEB (Mepis, Ubuntu, Knoppix, Debian...).
\item \alert{alien} : Convertir des paquetages RPM, DEV, Tar.Gz entre eux.
\item \alert{emerge} : Installer des ebuilds Gentoo.
\item \alert{up2date}, yum: Mise à jour sur Redhat.
\end{itemize}
\end{frame}

\begin{frame}{Recherche}
\begin{itemize}
\item \alert{whereis}, \alert{which} : Rechercher une application.
\item \alert{find} : Rechercher un dossier ou un fichier.
\item \alert{locate} : Idem mais dans une base de données. Plus rapide.
\item \alert{du} : Affiche la taille des répertoires.
\end{itemize}
\end{frame}

\begin{frame}{Utilisateurs, droits et permissions}
\begin{itemize}
\item \alert{adduser}, \alert{addgroup}, \alert{passwd}, \alert{usermod} : Gestion des utilisateurs, des groupes et des mots de passe.
\item \alert{su}, \alert{sudo} : Passer la main à l'administrateur (root) ou un autre utilisateur.
\item \alert{chmod} : Modifier les permissions sur un fichier.
\item \alert{chown} : Changer le propriétaire/groupe d'un fichier.
\item \alert{chgroup} : Changer le groupe d'un fichier.
\item \alert{chroot} : Modifier la racine. Parfois très utile.
\end{itemize}
\end{frame}

\begin{frame}{Gestion des processus / tâches}
\begin{itemize}
\item \alert{ps} , \alert{top} : Voir les processus en cours.
\item \alert{pidof} : Affiche le Process IDentifier d'une application.
\item \alert{kill}, \alert{killall} : Tuer un processus en cours.
\item \alert{fuser} : Afficher/tuer les processus accédant à un fichier/dossier.
\item \alert{nice} : Fixe la priorité d'exécution d'une tâche.
\item \alert{renice} : Modifier la priorité d'exécution d'une tâche en cours.
\item \alert{init} : Changer de "run level" : sortir de X, rebooter, éteindre.
\item \alert{pstree} : Afficher l'arbre généalogique des processus lancés.
\item \alert{cron} : Planifier des tâches périodiquement (outil crontab).
\item \alert{lsof} : listes les fichiers ouverts.
\end{itemize}
\end{frame}

\begin{frame}{Téléchargement}
\begin{itemize}
\item \alert{wget}, \alert{curl} : Outils permettant de télécharger les fichiers passés en paramêtre.
\item \alert{links} :\alert{ Navigateur} web en ligne de commande.
\end{itemize}
\end{frame}

\begin{frame}{Archivage}
\begin{itemize}
\item \alert{tar} : Rassembler dans un fichier une liste de fichiers ou de répertoires.
\item \alert{gzip}, \alert{bzip2} : Compresser un fichier au format gzip ou bzip2.
\item \alert{zip}, \alert{unzip} : Compresser un fichier au format zip.
\end{itemize}
\end{frame}
\begin{frame}{Enchaînements de commandes}
\begin{itemize}
\item \alert{>} \alert{>>} : Redigirer une sortie de commande vers un fichier.
\item \alert{;} \alert{||} \alert{\&\&} : Enchaîner des commandes séquentielle.
\item \alert{|} : Lancer des commandes simultanément via un 'pipe'.
\end{itemize}
\end{frame}

\begin{frame}{Variables d'environnement}
\begin{itemize}
\item \alert{export} : Charge une variable dans l'environnement courant.( export http\_proxy=http://monproxy :3128/ )
\item \alert{unset} : Supprime cette variable. ( unset http\_proxy )
\end{itemize}
\end{frame}

\begin{frame}{Administration du système}
\begin{itemize}
\item \alert{fdisk}, \alert{cfdisk} : Formater et partitionner un disque.
\item \alert{lsmod} : Afficher les modules du noyau présents en mémoire.
\item \alert{modprobe} : Manipulation de modules chargeables dans le noyau.
\item \alert{mount}, \alert{umount} : Monter/démonter des partitions dans l'arborescence de fichiers (/etc/fstab).
\item \alert{lspci}, \alert{lsusb} : Affiche les périphériques branchés sur ports PCI, AGP ou USB.
\item \alert{lshw} : Liste votre matériel de manière propre.
\item \alert{dmesg} Affiche les messages du noyau.
\end{itemize}
\end{frame}

\section{Conclusion}

\begin{frame}{Infos}

  Les sources de cette présentation sont sur le dépot :

  \begin{center}\url{github.com/bimade/introBash}\end{center}

  Cette présentation \emph{est} sous licence
  \href{http://creativecommons.org/licenses/by-nc-sa/4.0/}
  {Attribution-NonCommercial-ShareAlike 4.0 International (CC BY-NC-SA 4.0) }.

  \begin{center}\ccbyncsa\end{center}

\end{frame}

{\setbeamercolor{palette primary}{fg=black, bg=yellow}
\begin{frame}[standout]
  Questions?
\end{frame}
}

\appendix

\begin{frame}[fragile]{FAQ}
  A venir
\end{frame}

\begin{frame}[allowframebreaks]{References}

  \bibliography{demo}
  \bibliographystyle{abbrv}

\end{frame}

\end{document}