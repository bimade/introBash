\documentclass[10pt]{beamer}

\usetheme[progressbar=frametitle]{metropolis}
\usepackage{appendixnumberbeamer}
\usepackage[utf8]{inputenc} %Required for encoding text
\usepackage[T1]{fontenc}
\usepackage[french]{babel}
\usepackage{booktabs}
\usepackage[scale=2]{ccicons}

\usepackage{pgfplots}
\usepgfplotslibrary{dateplot}
\usepackage{xspace}

\usepackage{xcolor}
\usepackage{listings}
\usepackage{tcolorbox}
\tcbuselibrary{listings,skins}
%\lstset{basicstyle=\ttfamily\color{white},
%  showstringspaces=false,
%  xleftmargin = 0.5cm,
%  commentstyle=\color{red},
%  keywordstyle=\color{blue},
%  rulecolor=\color{white},
%  frameshape={RYR}{Y}{Y}{RYR},
%  backgroundcolor=\color{black},
%}
\lstdefinestyle{mystyle}{
     basicstyle=\ttfamily\color{white},
     language=bash,
     tabsize=1,
     keywordstyle=\color{blue}\bf,
     showstringspaces=false,
     morekeywords={mount, umount, ls, 
	 		cat, adduser, chown, chgrp, init, fuser}
 }
\newtcblisting{mylisting}{
      enhanced,                             %%% needed for shadow
      arc=2mm,
      top=0mm,
      bottom=0mm,
      left=0mm,
      right=0mm,
      boxrule=0pt,
      colback=black,
      %shadow={5mm}{-3mm}{0mm}{fill=black!50!white,opacity=0.5},             %%% here for shadow  and adjust as you like
      listing only,
      listing options={style=mystyle},
      hbox
}

\newcommand{\linux}{\textbf{\textsc{Linux}}\xspace}
\newcommand{\param}[1]{\textcolor{blue}{#1}}

\addtobeamertemplate{section page}{}{
  \centering
  \begin{minipage}{22em}
  \tableofcontents[sectionstyle=hide/hide,hideothersubsections]
  \end{minipage}\par
}
\title{Shell : commandes UNIX et programmation système}
\subtitle{Cours d'Outils de Programmation}
% \date{\today}
\date{}
\author{Bousbaa Imad Eddine}
\institute{Université des Sciences et de la Technologie de Houari Boumediène}
% \titlegraphic{\hfill\includegraphics[height=1.5cm]{logo.pdf}}

\begin{document}

\maketitle

\begin{frame}{Plan}
  \setbeamertemplate{section in toc}[sections numbered]
  \tableofcontents[hideallsubsections]
\end{frame}

\section{}

\begin{frame}[fragile]{Objectifs}
Apprendre à utiliser les commandes UNIX (bash).\\
Apprendre à automatiser des tâches en écrivant des scripts.\\

\end{frame}

\section{Shell}

\begin{frame}{Objectifs de shell}
\begin{enumerate}
\item Fournir une interface pour la saisie de commande
\item Redirection des entrées/sorties standards
\item Analyser les commandes
\begin{itemize}
	\item substitution de noms de fichiers
	\item substitution de variables
	\item redirection d'entrées/sorties
\end{itemize}
\item Exécution de commandes
\begin{itemize}
\item mode synchrone
\item mode asynchrone
\end{itemize}
\item Fournir un langage interprété
\end{enumerate}
\end{frame}


\begin{frame}{Type de shell}
\begin{itemize}
\item \alert{Bourne Shell}  \param{sh} : Shell disponible sur toute plateforme UNIX.
\item \alert{C shell}  \param{csh} : Shell développé par BSD.
\item \alert{Korn shell}  \param{ksh} : Bourne Shell étendu par l'AT\&T.
\item \alert{Bourne Again Shell} \param{bash} : Version améliorée de sh et csh. Fourni le plus souvent avec Linux.
\item \alert{Zero Shell}  \param{zsh} : shell avec beaucoup de fonctionnalités : typage, substitution et complétion très poussées.
\item \alert{Tenex} \param{tcsh}  : csh étendu
\item \alert{rc} \param{rc}  : Implémentation pour UNIX du shell de Plan 9
\end{itemize}
\end{frame}

\begin{frame}{Utilisation du shell}
Deux modes d'utilisation :
\begin{itemize}
\item interactif : en ligne de commande.
\begin{enumerate}
  \item Présente une invite (prompt) à l'utilisateur et attend que celui-ci tape une commande ;
  \item Exécute * la commande tapée par l'utilisateur
  \item Retour en $1$.
\end{enumerate}
\item non-interactif : scripts shell, batch
  \begin{enumerate}
  \item Lit une ligne du fichier
  \item Exécute * les instructions données dans la ligne du fichier
  \item Passe à la ligne suivante
  \item Retour en $1$
  \end{enumerate}
\end{itemize}

Le programme s'arrête lorsqu'il n'y a plus de ligne à lire ou lorsqu'un instruction spéciale (exit ou return) est rencontrée.

\end{frame}

\section{Bash}
\subsection{Les consoles}
\begin{frame}[fragile]{c'est quoi une console ?}

La console est une interface entre votre système et vous. Elle s'appuie sur un
interpréteur de commandes (shell en anglais), sorte de langage. Il en existe
plusieurs dans le monde UNIX mais le plus utilisé s'appelle Bash. Il est installé par
défaut sur probablement toutes les distributions Linux et contient des centaines de
commandes différentes.

\end{frame}


\begin{frame}[fragile]{Les consoles}

Bash peut être utilisé directement en ligne de commande, c'est-à-dire en dehors de
toute interface graphique (KDE, Gnome etc). Il existe aussi des interfaces graphiques. Il en existe de nombreuses : Gnome-Terminale, konsole, aterm, eterm, xterm... Leurs différences étant cosmétiques car elles utilisent finalement le même interpréteur de commande, Bash.

Vous accédez ainsi à Bash si, après le démarrage du noyau Linux, vous ne vous connectez pas à un environnement graphique (KDE, Gnome...). Sa puissance est alors à son maximum car le processeur ne s'occupe pas de gérer des fenêtres. 

Pour ouvrir une console, le plus simple est d'explorer le menu de l'environnement graphique à sa recherche ou bien de cliquer l'icône probablement installée dans la barre des tâches.
\end{frame}


\begin{frame}{Les terminaux virtuels}

Que vous soyez connecté à une interface graphique ou non, il existe un moyen
pratique d'ouvrir rapidement une ligne de commande supplémentaire. Le raccourci
clavier est CRTL-ALT-Fx. Remplacez le x par un chiffre de 1 à 7.

Exemple : \alert{CRTL-ALT-F1}. 

Les 6 premiers ouvrent des terminaux virtuels (auxquels il faudra vous connecter : identifiant et mot de passe) et le 7èm ramène à l'interface graphique (si ouverte). C'est surtout pratique quand aucune interface graphique (KDE, Gnome...) n'est ouverte, vous pouvez ainsi lancer en parallèle des commandes.
\end{frame}

\begin{frame}{Le prompt}
	Le prompt est une invite de commande. C'est le message que la console place à
chaque début de ligne, en attendant votre commande.

Exemple : \\
\alert{luteola@localhost \textasciitilde \$}


Ceci indique que le simple utilisateur (symbole \$) 'luteola' travaille sur l'ordinateur
'localhost' est qu'il est dans son répertoire personnel (symbole \textasciitilde) : /home/luteola.


\alert{root@localhost /bin \#}
Dans cet exemple, l'administrateur (root et symbole \#) travaille dans le dossier
/bin.
\end{frame}


\begin{frame}{Sensibilité à la casse}
La console est sensible à la casse (majuscule / minuscule) ! Ainsi, ls est une
commande alors que LS ne veut rien dire. Idem pour l'écriture des dossiers et
fichiers. La moindre faute ou espace mal placé aura des conséquences.
\end{frame}


\begin{frame}{Écrire des chemins d'accès.}
	La structure des répertoires sous Linux est pyramidale, la pointe étant appelée la
racine. 

En console, on peut bien sûr se déplacer dans ces répertoires.


Pour séparer des répertoires dans un chemin, on utilise des slashs /, comme on le
ferait pour une adresse Internet en fait. N'utilisez donc pas les anti-slash, comme
vous le demanderai Windows. 

Exemple : /home/george/musique.\pause

Les anti-slash servent plutôt à insérer (on dit 'protéger') des caractères spéciaux ou
des espaces. Créez 'à la souris' un répertoire "Nouveau Dossier" sur votre bureau
(clic droit, Nouveau...) puis en console tapez cd \textasciitilde/Desktop/ et utilisez la touche
Tabulation (qui autocomplète un chemin). 'Nouveau' et 'dossier' sont séparés par
un anti-slash.
\end{frame}


\begin{frame}{Autocomplétion des commandes}
	Bash peut "deviner" la fin d'un chemin ou d'une commande. Ceci s'opère par la touche 'tabulation'. Très pratique pour accélérer la saisie de commandes ou retrouver une syntaxe ! 
	
	Exemples :

	Tapez up et appuyez sur 'tabulation'. Bash affiche alors la liste des commandes commençant par 'up'.

	Tapez cd /bi puis 'tabulation'. Bash complète la fin du chemin s'il existe, cd /bin/ en l'occurrence.

	Mais encore plus fort avec "bash\_completion", il peut même compléter lors de l'installation de logiciels.

	Par exemple sur Ubuntu ou Debian si vous tapez "source /etc/bash\_completion" et qu'en suite vous tapez "aptitude install fire" puis la touche "tabulation", il va vous donné tous les logiciels disponibles qui commencent par "fire"
\end{frame}


\begin{frame}{Historique des commandes}

En utilisant les flèches haut et bas, vous parcourez l'historique des commandes tapées sous votre login.

Ou avec la commande "history" il peut vous retrouver des commandes que vous
avez tapez le matin ou la veille.


Exemple : \$ \alert{history} | \alert{grep} cat

\alert{CTRL-R} permet également de faire une recherche rapide sur une commande.
\end{frame}

\begin{frame}{jokers}
Vous pouvez utiliser des jokers dans une commande :
\begin{itemize}
\item ? représente n'importe quel caractère (a, Y, 3, @...).
\item * représente n'importe quel chaîne de caractères (un mot, une phrase...).
\item Exemple : ls photo* affiche la liste des fichiers ou dossiers commençant par
\item le mot photo.
\item *[a]* représente une chaîne contenant la lettre a.
\item *[!a]* représente une chaîne ne contenant pas la lettre a.
\item [abc]* représente une chaîne commençant par la lettre a,b ou c.
\item [a-d]* représente une chaîne commençant par la lettre a, b, c ou d.
\end{itemize}
\end{frame}
\section{Les principales commandes console (BASH) sous Linux}

\begin{frame}[fragile]{syntaxe d'une commande}
la syntaxe classique d'une commande Unix :

\begin{center}
cmd [-opt1] [-opt2] ... [--] arg1 arg2 ...
\end{center}

où :

\begin{itemize}
\item  "cmd" correspond au nom de la commande,
\item  "-opt" correspond à une option possible,
\item  "arg" correspond à un argument.
\end{itemize}
\pause

Notons que les arguments sont le plus souvent un nom de fichier que la commande manipule. On peut utiliser "--" pour séparer explicitement les options des arguments quand la commande est ambiguë.

Pour obtenir l'aide en ligne sur une commande, il suffit d'utiliser le "man". Pour obtenir l'aide sur la commande "cmd", il suffit de taper :
\begin{mylisting}
man cmd # Afficher le manuel de la commande cmd
\end{mylisting}
\end{frame}

\begin{frame}[fragile]{Type de commmande}
Les commandes internes sont réalisées de manière interne par le shell lui-même ; c'est-à-dire qu'il n'y a pas de création de processus
pour exécuter la commande. Ces commandes ne possèdent pas d'exécutables associés puisqu'elles sont codées en interne au shell.

Une méthode pour identifier les builtins est d'utiliser la commande interne type.
\begin{mylisting}
$ type cd
cd is a shell builtin
$ type echo
echo is a shell builtin
$ type ls
ls is /bin/ls
$ type cat
cat is /bin/cat
$ type gcc
gcc is /usr/bin/gcc
\end{mylisting}
\end{frame}

\begin{frame}{Commandes}
Voici une liste des commandes console Bash les plus utilisées en routine. Certaines
plus que d'autres. N'oubliez pas que souvent des interfaces graphiques existent
pour ces commandes.
\end{frame}
\begin{frame}{Commandes de base }
\begin{itemize}
\item \alert{ls} : Afficher la liste des dossier et fichiers d'un répertoire.
\item \alert{cd} : Se déplacer vers un répertoire.
\item \alert{pwd} : Affiche le nom du répertoire courant.
\item \alert{mkdir} : Créer un répertoire
\item \alert{rmdir} : Supprimer un répertoire.
\item \alert{rm} : Effacer un fichier ou un répertoire.
\item \alert{cp} : Copier un dossier ou un répertoire.
\item \alert{mv} : Déplacer (renommer) un dossier ou un répertoire.
\item \alert{ln} : Créer un lien vers un fichier.
\item \alert{touch} : Permet de changer la date d'un fichier à l'heure courante ou de créer un fichier s'il n'existe pas (touch /home/user/test).
\item \alert{alias} : Créer un alias d'une commande ( exemple : alias rm="rm -i" )
\end{itemize}
\end{frame}

\begin{frame}{Manipulations de texte}
\begin{itemize}
\item \alert{cat} : Afficher le contenu d'un fichier texte. Concaténer des fichiers.
\item \alert{more}, less : Afficher le contenu d'un fichier texte intelligemment.
\item \alert{grep} : Rechercher une chaîne de caractère.
\item \alert{sed} : Remplacer des occurrences de texte dans des fichiers.
\item \alert{echo} : Envoyer un message texte. Par exemple vers un fichier.
\item \alert{tail} : N'affiche que les dernières lignes d'un fichier (-n permet de spécifier le nombre de lignes à afficher).
\item \alert{head} : Comme tail, mais affiche les N premières lignes d'un fichier (N=10 par défaut).
\item \alert{sort} : Trier un fichier texte.
\item \alert{uniq} : Supprime ou compte les doublons sur un fichier trier.
\item \alert{wc} : Compte le nombre de lignes, mots et caractères.
\item \alert{awk} : Remise en forme de données texte.
\end{itemize}
\end{frame}

\begin{frame}{Installer des logiciels}
\begin{itemize}
\item \alert{urpmi} : Installer des paquetages RPM sous Mandriva.
\item \alert{apt-get}, \alert{aptitude} : Installer des paquets DEB (Mepis, Ubuntu, Knoppix, Debian...).
\item \alert{alien} : Convertir des paquetages RPM, DEV, Tar.Gz entre eux.
\item \alert{emerge} : Installer des ebuilds Gentoo.
\item \alert{up2date}, yum: Mise à jour sur Redhat.
\end{itemize}
\end{frame}

\begin{frame}{Recherche}
\begin{itemize}
\item \alert{whereis}, \alert{which} : Rechercher une application.
\item \alert{find} : Rechercher un dossier ou un fichier.
\item \alert{locate} : Idem mais dans une base de données. Plus rapide.
\item \alert{du} : Affiche la taille des répertoires.
\end{itemize}
\end{frame}

\begin{frame}{Utilisateurs, droits et permissions}
\begin{itemize}
\item \alert{adduser}, \alert{addgroup}, \alert{passwd}, \alert{usermod} : Gestion des utilisateurs, des groupes et des mots de passe.
\item \alert{su}, \alert{sudo} : Passer la main à l'administrateur (root) ou un autre utilisateur.
\item \alert{chmod} : Modifier les permissions sur un fichier.
\item \alert{chown} : Changer le propriétaire/groupe d'un fichier.
\item \alert{chgroup} : Changer le groupe d'un fichier.
\item \alert{chroot} : Modifier la racine. Parfois très utile.
\end{itemize}
\end{frame}

\begin{frame}{Gestion des processus / tâches}
\begin{itemize}
\item \alert{ps} , \alert{top} : Voir les processus en cours.
\item \alert{pidof} : Affiche le Process IDentifier d'une application.
\item \alert{kill}, \alert{killall} : Tuer un processus en cours.
\item \alert{fuser} : Afficher/tuer les processus accédant à un fichier/dossier.
\item \alert{nice} : Fixe la priorité d'exécution d'une tâche.
\item \alert{renice} : Modifier la priorité d'exécution d'une tâche en cours.
\item \alert{init} : Changer de "run level" : sortir de X, rebooter, éteindre.
\item \alert{pstree} : Afficher l'arbre généalogique des processus lancés.
\item \alert{cron} : Planifier des tâches périodiquement (outil crontab).
\item \alert{lsof} : listes les fichiers ouverts.
\end{itemize}
\end{frame}

\begin{frame}{Téléchargement}
\begin{itemize}
\item \alert{wget}, \alert{curl} : Outils permettant de télécharger les fichiers passés en paramêtre.
\item \alert{links} :\alert{ Navigateur} web en ligne de commande.
\end{itemize}
\end{frame}

\begin{frame}{Archivage}
\begin{itemize}
\item \alert{tar} : Rassembler dans un fichier une liste de fichiers ou de répertoires.
\item \alert{gzip}, \alert{bzip2} : Compresser un fichier au format gzip ou bzip2.
\item \alert{zip}, \alert{unzip} : Compresser un fichier au format zip.
\end{itemize}
\end{frame}
\begin{frame}{Enchaînements de commandes}
\begin{itemize}
\item \alert{>} \alert{>>} : Redigirer une sortie de commande vers un fichier.
\item \alert{;} \alert{||} \alert{\&\&} : Enchaîner des commandes séquentielle.
\item \alert{|} : Lancer des commandes simultanément via un 'pipe'.
\end{itemize}
\end{frame}

\begin{frame}{Variables d'environnement}
\begin{itemize}
\item \alert{export} : Charge une variable dans l'environnement courant.( export http\_proxy=http://monproxy :3128/ )
\item \alert{unset} : Supprime cette variable. ( unset http\_proxy )
\end{itemize}
\end{frame}

\begin{frame}{Administration du système}
\begin{itemize}
\item \alert{fdisk}, \alert{cfdisk} : Formater et partitionner un disque.
\item \alert{lsmod} : Afficher les modules du noyau présents en mémoire.
\item \alert{modprobe} : Manipulation de modules chargeables dans le noyau.
\item \alert{mount}, \alert{umount} : Monter/démonter des partitions dans l'arborescence de fichiers (/etc/fstab).
\item \alert{lspci}, \alert{lsusb} : Affiche les périphériques branchés sur ports PCI, AGP ou USB.
\item \alert{lshw} : Liste votre matériel de manière propre.
\item \alert{dmesg} Affiche les messages du noyau.
\end{itemize}
\end{frame}

\section{Programmation système}
\subsection{Mon premiers script}
\begin{frame}[fragile]{script}
Un script est une suite d'instructions élémentaires qui sont exécutées de façon séquentielle (les unes après les autres) par le langage de script. Dans notre cas, on parlera du shell Bash.
\pause 


Pour commencer, il faut savoir qu'un script est un \alert{fichier texte} standard pouvant être créé par n'importe quel éditeur : vi, emacs, kedit, gnotepad, ou autre. 

D'autre part, conventionnellement, un script commence par une ligne de commentaire contenant le nom du langage à utiliser pour interpréter ce script, soit dans notre
cas : /bin/bash (on parle alors de "script shell").

Nous allons donc commencer par écrire un simple script que l'ont va nommer
"bonjour\_monde" et qui nous affichera "Bonjour monde !" lors de son éxecution :

\begin{mylisting}
#!/bin/sh
echo "Bonjour, Monde !"
echo "Un premier script est ne"
\end{mylisting}

\end{frame}


\begin{frame}[fragile]{Execution}
\begin{mylisting}
[user@becane user]$ sh bonjour_monde
\end{mylisting}

Ou bien on peut le rendre exécutable avec la commande chmod comme ceci :
\begin{mylisting}
[user@becane user]$ chmod +x bonjour_monde
\end{mylisting}
Des lors nous pourrons exécuter notre script de cette façon :
\begin{mylisting}
[user@becane user]$ ./bonjour_monde
\end{mylisting}
\end{frame}
\subsection{D. Quelques conseils concernant les commentaires}
\begin{frame}[fragile]{Quelques conseils concernant les commentaires}

Dans un script shell, est considéré comme un commentaire tout ce qui suit le
caractère \# et ce, jusqu'à la fin de la ligne. Usez et abusez des commentaires :
lorsque vous relirez un script six mois après l'avoir écrit, vous serez bien content de
l'avoir documenté. 

Un programme n'est jamais trop documenté, en revanche il peut
être mal documenté ! Un commentaire est bon lorsqu'il décrit pourquoi on fait
quelque chose, pas quand il décrit ce que l'on fait.


Exemple :
\begin{mylisting}
#!/bin/sh
# pour i parcourant tous les fichiers,
for i in ./* ; do
# copier le fichier vers .bak
cp $i $i.bak
# fin pour
done
\end{mylisting}

\end{frame}

\begin{frame}[fragile]{Les commentaires}

Que fait le script ? Les commentaires ne 
l'expliquent pas ! Ce sont de mauvais
commentaires.
En revanche :
\begin{mylisting}
#!/bin/sh
# on veut faire une copie de tous les 
# fichiers du repertoire courant
for i in ./* ; do
# sous le nom *.bak
cp $i $i.bak
done
\end{mylisting}

Là au moins, on sait ce qu'il se passe (il n'est pas encore important de connaître les
commandes de ces deux fichiers).
\end{frame}

\begin{frame}[fragile]{Caractères spéciaux}

\begin{itemize}
\item \alert{tabulation, espace} : Délimiteur de mot
\item \alert{retour chariot }: Fin de la commande à exécuter
\item \alert{ \& }: Lance une commande en tâche de fond
\item \alert{ ;;; }: Séparateur de commande
\item \alert{* ? [] [ \textasciicircum ] }: Substitution de noms de fichiers
\item \alert{\&\& ||! }: Opérateurs booléens
\item \alert{'” \ }: Caractères de quotation
\item \alert{<> << >> | }: Opérateurs de redirection d'entrées sorties 
\item \alert{\$ } : Valeur d'une variable 
\item \alert{\# } : Début de commentaires
\item \alert{() \{\} } : Groupement de commande
\end{itemize}
\end{frame}


\begin{frame}{Caractères de substitution}
\begin{itemize}
\item \alert{*} : N'importe quelle séquence de caractères
\item \alert{?} : N'importe quel caractère
\item \alert{[]} : N'importe quel caractère choisi dans les caractères donnés entre crochets
\item \alert{[ \textasciicircum ]} : n'importe quel caractère sauf ceux dans les caractères donnés entre crochets
\item \alert{[-]} : n'importe quel caractère dans la plage de caractères donnés entre crochets
\end{itemize}
\end{frame}


\begin{frame}{Mots résevés}
\begin{itemize}
\item \alert{case, in, esac}
\item \alert{for, in, do, done}
\item \alert{if, then, elif , else, fi}
\item \alert{while, do, done}
\item \alert{until, do, done}
\item \alert{break, continue}
\item \alert{return, exit}
\end{itemize}
\end{frame}

\begin{frame}{Variables}

Une variable est une suite de caractères alphanumérique associée éventuellement à une valeur.

Une variable doit être de la forme :
 \begin{center}
 [A-Za-z\_][A-Za-z\_]*,
 \end{center}
 par exemple, toto, TOTO, MA\_Valeur, t, sont des variables.

 \pause
 La notion de typage n'existe pas en shell, toutes les valeurs sont des chaînes de caractères, et par conséquent :
 \begin{itemize}
 \item il n'y a pas de déclaration de variables.
 \item il n'y a pas de valeur numérique.
 \end{itemize}
 \end{frame}

\begin{frame}{Variables : assignation et utilisation}

\begin{itemize}
\item Assignation de variable
\begin{center}
variable=valeur
\end{center}
\item Consultation de variable :
\begin{center}
\$variable ou \${variable}
\end{center}
\item Visualisation de toutes les variables définies :
\begin{center}
set
\end{center}
\end{itemize}
\end{frame}

\begin{frame}{Variables : destruction et protection}
\begin{itemize}
\item Destruction d’une variable :
\begin{center}
\alert{unset} variable
\end{center}
\item Protection d’une variable en écriture :
\begin{center}
\alert{readonly} variable
\end{center}
\item Liste des variables protégées en écriture : \alert{readonly}
\end{itemize}
\end{frame}

\begin{frame}[fragile]{Précaution sur les variables}
Mieux vaut utiliser \${nom\_variable} que \$nom\_variable :

\begin{mylisting}
$ MOT=volume
$ echo $MOT42

$ echo ${MOT}42
volume42
$
\end{mylisting}
\end{frame}

\begin{frame}[fragile]{Opérations numériques}

Il n’y a pas de type ”numérique en shell”

$ titi=3
$ echo $titi
3
$ titi=$titi + 1
+ : not found
$ titi=$titi+1
$ echo $titi
3+1
$
Il faut utiliser des commandes spécifiques pour (( utiliser )) les chaînes à valeur numérique (cf. \alert{expr})

\end{frame}

\begin{frame}[fragile]{variables d'environnement : export}

Une variable définie n’est accessible que dans le shell courant. Elle n’est pas transmise aux processus fils. Pour qu’elle soit transmise aux processus fils, il faut utiliser le mot-clef export variable

Soit le script affiche\_toto :

echo "Le contenu de la variable TOTO est : \$TOTO"

\begin{mylisting}
$ TOTO=42
$ echo $TOTO
42
$ ./affiche_toto
Le contenu de la variable TOTO est :
$ export TOTO
$ ./affiche_toto
Le contenu de la variable TOTO est : 42
$
$
\end{mylisting}

Une variable exportée est appelée \alert{variable d’environnement}

\end{frame}

\begin{frame}{Variables d'environnement}

Variables d’environnement courantes utilisées par de nombreuses commandes (ainsi que par le shell).
\begin{itemize}
\item \alert{HOME} : Le répertoire (( maison )) de l’utilisateur 
\item \alert{USER} : Le login utilisateur
\item \alert{SHELL} : Le shell utilisateur
\item \alert{PATH} : La liste des chemins dans laquelle le shell cherchera les commandes exécutables qui seront tapés sans indication de chemin
\item \alert{TERM} : Le type de terminal
\item \alert{PS1} : L’invite (prompt)
\item \alert{PWD} : Le répertoire courant
\item \alert{LANG} : Le langage utilisé
\end{itemize}
L’affichage des variables d’environnement se fait par la commande \alert{env}.
\end{frame}

\begin{frame}{Substitution de variables}
\begin{itemize}
\item \alert{\${variable:-valeur}} : utilise la valeur de la variable si elle est positionnée et est non nulle, sinon utilise la valeur donnée.
 \item \alert{\${variable:=valeur}} : utilise la valeur de la variable si elle est positionnée et est non nulle, sinon affecte la valeur à la variable et utilise la valeur donnée.
 \item \alert{\${variable:?valeur}} : utilise la valeur de la variable si elle est positionnée et est non nulle, sinon écrit la valeur sur l’erreur standard et fait un exit avec un statut non nul
 \item \alert{\${variable:+valeur}} : utilise une chaîne vide si la variable est positionnée et est non nulle, sinon, utilise la valeur.
 \item \alert{\${\#variable}} : utilise la taille de la valeur de la variable
\end{itemize}
\end{frame}


 \begin{frame}{Variables prédéfinies}
 
 \begin{itemize}
 \item \alert{\$?} : Code de retour de la dernière commande exécutée
 \item \alert{\$\$} : PID du shell en train de s’exécuter
 \item \alert{\$!} : PID de la dernière commande lancée en tâche de fond 
 \item \alert{\$-} : options passées au shell courant
 \end{itemize}
 \end{frame}

\begin{frame}{Caractères d'échapement}
’ " $\backslash$ changent la façon dont le shell interprête les caractères spéciaux
\begin{itemize}
\item \alert{’ (single-quote)} : le shell ignore tout caractère spéciaux entre deux ’
\item \alert{” (double-quote)} : le shell ignore tout caractère spéciaux entre deux ”, à l’exception de \$ et $\backslash$ et ‘
\item \alert{\ (antislash ou le backslash)} : shell ignore le caractère spécial suivant le $\backslash$
\item \alert{‘ (backquote ou antiquote)} : le shell exécute ce qu’il y a entre deux ‘
\end{itemize}
\end{frame}

\begin{frame}{Variables de passage de paramètre à un script (variables positionelles)}
Ces variables sont automatiquement affectées lors d'un appel de script suivi d'une liste de paramètres. Leurs valeurs sont récupérables dans \$1, \$2 ...\$9.

\begin{itemize}
\item \alert{\$*} : Tous les paramètres passés au script shell sous la forme de mots séparés
\item \alert{\$@} : Tous les paramètres passés au script shell
\item \alert{\$\#} : Le nombre de paramètres passés au script shell
\item \alert{\$0} : Nom du script shell
\item \alert{\$1, \$2, ...} : Les paramètres passés dans l'ordre.
\end{itemize}
\end{frame}

\begin{frame}[fragile]{Exemple}
\begin{mylisting}
# !/bin/sh
# Mon programme qui affiche les parametres de 
# la ligne de commande
echo "* Le nom du programme est : $0"
echo "* Le troisieme parametre est : $3"
echo "* Le nombre de parametre est : $#"
echo "* Tous les parametres (mots individuels) : $*"
echo "* Tous les parametres : $@"
exit 0
\end{mylisting}
\end{frame}
\begin{frame}[fragile]{Exemple}
\begin{mylisting}
$ ./script2.sh un "deux" "trois quatre" cinq
* Le nom du programme est : ./script2.sh
* Le troisieme parametre est : trois quatre
* Le nombre de parametre est : 4
* Tous les parametres (mots individuels) : un deux 
trois quatre cinq
* Tous les parametres : un deux trois quatre cinq
$
$
\end{mylisting}
\end{frame}
\begin{frame}[fragile]{Passage de paramètres : précaution}
Considérons le script suivant :
\begin{mylisting}
# !/bin/sh
echo $1 $2 $3 $4 $5 $6 $7 $8 $9 $10 $11 $12
exit 0
\end{mylisting}

\pause
\begin{mylisting}
$ ./script3 faux.sh un deux trois quatre cinq six
sept huit neuf dix onze douze
un deux trois quatre cinq six sept huit neuf un0 un1
un2
$
\end{mylisting}
\end{frame}

\begin{frame}[fragile]{}
\begin{mylisting}
# !/bin/sh
echo $1 $2 $3 $4 $5 $6 $7 $8 $9 ${10} ${11} ${12}
exit 0
\end{mylisting}
\pause
\begin{mylisting}
$ ./script3 vrai.sh un deux trois quatre cinq six
sept huit neuf dix onze douze
un deux trois quatre cinq six sept huit neuf dix onze
douze
$
\end{mylisting}
\end{frame}

\subsection{tests}
\begin{frame}[fragile]{Commandes de test : test, [ }
\alert{test} expression ou [ expression ] permet d’évaluer une expression.

\begin{itemize}
\item Si vrai, renvoie 0, sinon, renvoie 1.
\item S’il n’y a pas d’expression, renvoie 1 (false).
\end{itemize}

\pause

\begin{itemize}
\item \alert{-d fic} vrai si le fichier existe et est un répertoire
\item \alert{-f fic} vrai si le fichier existe et est un fichier (ordinaire)
\item \alert{-h fic} : vrai si le fichier existe et est un lien symbolique
\item \alert{-x fic} : vrai si le fichier existe et est autorisé en exécution
\item \alert{-w fic} : vrai si le fichier existe et est autorisé en écriture
\item \alert{-r fic} : vrai si le fichier existe et est autorisé en lecture
\item \alert{ch1 = ch2} : si les deux chaines sont identiques
\item \alert{ch1 != ch2} : si les deux chaines sont différentes
\end{itemize}
\end{frame}

\begin{frame}{tests}
\begin{itemize}
\item \alert{n1 -eq n2} : si les deux nombres sont numériquement égaux
\item \alert{n1 -neq n2} : si les deux nombres sont numériquement inégaux
\item \alert{n1 -lt n2} : si n1 est numériquement inférieur à n2
\item \alert{n1 -gt n2} : si n1 est numériquement supérieur à n2
\item \alert{n1 -le n2} : si n1 est numériquement inférieur ou égal à n2
\item \alert{n1 -ge n2} : si n1 est numériquement supérieur ou égal à n2
\item \alert{! exp1} : vrai si l’expression est fausse (et vice-versa)
\item \alert{exp1 -a exp2} : vrai si les deux expressions sont vraies (AND)
\item \alert{exp1 -o exp2} : vrai si au moins une expression est vraie (OR)
\end{itemize}

\end{frame}


\begin{frame}[fragile]{tests : exemple}
\begin{mylisting}
$ test -f fic-qui-existe
$ echo $?
0
$ test -f fic-qui-n-existe-pas
$ echo $?
1
$ [ 3 -lt 42 -a -x fic-executable ]
$ echo $?
0
$ [ 3 -gt 42 -a -x fic-executable ]
$ echo $?
1
$
$
\end{mylisting}
\end{frame}


\begin{frame}[fragile]{Valeur de retour}
L’exécution d’une commande modifie la variable de retour \$?. Des commandes telles que ls, cd, rm, ... modifieront la variable de retour \$? suivant que l’opération se soit bien passée ou non.

Note: \alert{:} est une commande interne retournant toujours vrai (0).
\pause

\begin{mylisting}
ls wzwzwzw
ls : zwzwz : No such file or directory
$ echo $?
1
$ ls fichier
fichier
$ echo $?
0
$ :
$ echo $?
0
\end{mylisting}
\end{frame}

\begin{frame}[fragile]{Listes de commandes}

Une liste de commandes peut être écrite de deux manières différentes :
\begin{itemize}
\item Une séquence de commande s’écrira : commande1 ; commande 2 ; \ldots
\item ou
\end{itemize}
\begin{mylisting}
commande1
commande2
\end{mylisting}

\pause 
Attention, chaque nouvelle commande (non interne) est exécutée dans un nouveau processus.

Sous-shell : (commande1 ; commande2)

Exemple : (cd repertoire ; touch toto)
exécute la commande touch toto sans changer la valeur du répertoire initial.

\end{frame}

\subsection{Syntaxe algorithmique}
\begin{frame}[fragile]{if-then-elif-else-fi}
La syntaxe des bloques conditionnels 
\begin{mylisting}
if liste-commandes-1
then liste-commandes-2
elif liste-commandes-3
else liste-commandes-4
fi
\end{mylisting}

\pause
Exemple :
\begin{mylisting}
if ls monfichier
then echo "la commande ls monfichier a reussi"
else echo "la commande ls monfichier a echoue"
fi
\end{mylisting}
\end{frame}

\begin{frame}[fragile]{if-then-elif-else-fi : exemple}

Autre exemple :
\begin{mylisting}
if [ -d toto ]
then echo "toto est un repertoire"
elif [ -h toto ]
then echo "toto est un lien symbolique"
else echo "faut pousser l'investigation plus loin"
fi
\end{mylisting}

\end{frame}

\begin{frame}[fragile]{Opérateurs logiques : \&\&, ||}
\&\&, || sont souvent utilisé pour une forme compacte et élégante.
\pause

Le ET logique :

\begin{center}
cmd1 \&\& cmd2 = if cmd1 then cmd2 fi
\end{center}
Si la commande 1 réussi, alors faire la commande 2.

\pause
exemple: gcc toto.c -o toto \&\& ./toto

\pause
Le OU logique :
\begin{center}
cmd1 || cmd2 = if ! cmd1 then cmd2 fi
\end{center}
Si la commande 1 a échouée, alors faire la commande 2.

exemple : cat toto || touch toto
\end{frame}

\begin{frame}[fragile]{Branchement conditionnel : case-esac}
Le branchement conditionnel \alert{case-esac} exécute la liste-commandes suivant le motif reconnu. Sa syntaxe est :

\alert{case} valeur \alert{in} \\
\param{motif)} liste-commandes ;; \\
\alert{esac} \\

Le motif à reconnaitre peut s’exprimer sous forme d’expression rationnelle utilisant les métacaractères : \param{*?[]!-|}.
\end{frame}

\begin{frame}[fragile]{Branchement conditionnel : case-esac}

Exemple :

\begin{mylisting}
case $reponse in
[Yy ][eE][sS] | [oO][uU][iI] | OK ) echo "Tu approuves" ;;
[Nn][oO]*) echo "Tu desapprouves" ;;
bof) echo "C'est tout comme" ;;
pfff*) echo "Pas la peine de repondre" ;;
*) echo "reponse idiote" ;;
esac
\end{mylisting}
\end{frame}

\begin{frame}{Boucle for-do-done}

\alert{for} \param{variable} \alert{in} \param{liste de mots}\\
\alert{do}\\
liste-commandes \\
\alert{done}

La variable prend successivement les valeurs de la liste de mots, et
pour chaque valeur, liste-commandes est exécutée.

\end{frame}

\begin{frame}{while-do-done}

\alert{while} liste-commandes-1 \\
\alert{do} \\
liste-commandes-2 \\
\alert{done}\\

La valeur testée par la commande while est l’état de sortie de la dernière commande de liste-commandes-1. Si l’état de sortie est 0,
alors le shell exécute liste-commandes-2 puis recommence la boucle.
\end{frame}
\begin{frame}{Until-do-done}

\alert{until} liste-commandes-1\\
\alert{do}\\
liste-commandes-2\\
\alert{done}\\

Le shell exécute liste-commandes-2 puis teste l’état de sortie de liste-commandes-1. Si l’état de sortie est 0, alors, la boucle est recommencée.
\end{frame}

\begin{frame}{Contrôle du flux d’exécution : break, continue}
  
\begin{alertblock}{break ou break n}
  permet de sortir d’une boucle for, while ou until. Si n est précisé, il indique le nombre d’imbrication concernée par le break.
\end{alertblock}

\pause

\begin{alertblock}{continue n}
permet de passer à l’itération suivante d’une boucle for, while ou until. Si n est précisé, il indique le nombre d’imbrication concernée par le continue.
\end{alertblock}

\end{frame}

\begin{frame}{Les fonctions}

  On peut regrouper les commandes au sein d’une fonction. Une fonction se définit de la manière suivante :
  \begin{alertblock}
nom\_fonction ()\\
\{\\
liste-commandes \\
\}\\
  \end{alertblock}
Les paramètres au sein de la fonction sont accessibles via \$1, \$2,... \$@, \$\#. 

L’appel d’une fonction se fait de la manière suivante :
\begin{center}
nom\_fonction parametre1 parametre2...
\end{center}

Une fonction doit être déclarée avant de pouvoir être exécutée.
\end{frame}

\begin{frame}{Code de retour : return, exit}
  
\begin{alertblock}{return n}
Renvoie une valeur de retour pour la fonction shell.
\end{alertblock}

\begin{alertblock}{exit n}
Provoque l’arrêt du shell courant avec un code retour de n si celui-ci est spécifié. S’il n’est pas spécifié, il s’agira de la valeur de retour de la dernière commande exécutée.
\end{alertblock}
\end{frame}


\section{Conclusion}

\begin{frame}{Infos}

  Les sources de cette présentation sont sur le dépot :

  \begin{center}\url{github.com/bimade/introBash}\end{center}

  Cette présentation \emph{est} sous licence
  \href{http://creativecommons.org/licenses/by-nc-sa/4.0/}
  {Attribution-NonCommercial-ShareAlike 4.0 International (CC BY-NC-SA 4.0) }.

  \begin{center}\ccbyncsa\end{center}

\end{frame}

{\setbeamercolor{palette primary}{fg=black, bg=yellow}
\begin{frame}[standout]
  Questions?
\end{frame}
}

\appendix

\begin{frame}[fragile]{FAQ}
  A venir
\end{frame}

\begin{frame}[allowframebreaks]{References}

  \bibliography{demo}
  \bibliographystyle{abbrv}

\end{frame}

\end{document}